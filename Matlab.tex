% .:: Laden der LaTeX4EI Formelsammlungsvorlage
\documentclass[fs]{latex4ei}


\usepackage{listings}
\definecolor{listinggray}{gray}{0.9}
\definecolor{lbcolor}{rgb}{0.9,0.9,0.9}
\lstset{
    backgroundcolor=\color{lbcolor},
    basicstyle=\footnotesize\tt,
    tabsize=4,
    numbers=none,				% switch numbers on: left
    numberstyle=\tiny\sf,
    numbersep=1em,
    language=Matlab,
    %upquote=true,
    %aboveskip={1\baselineskip},
    %belowskip={0\baselineskip},
    abovecaptionskip={\baselineskip},
    belowcaptionskip={0\baselineskip},
    columns=fixed,
    showstringspaces=false,
    extendedchars=true,
    linewidth={6.2cm},
    xleftmargin={1em},
    framexleftmargin={1pt},
    framexrightmargin={10pt},
    framextopmargin={1pt},
    framexbottommargin={1pt},
    breaklines=true,
    prebreak = \raisebox{0ex}[0ex][0ex]{\ensuremath{\hookleftarrow}},
    frame=single,
    showtabs=false,
    showspaces=false,
    showstringspaces=false,
    identifierstyle=\ttfamily,
    %tagstyle=\bf,
    keywordstyle=\color[rgb]{0,0,1},
    commentstyle=\color[rgb]{0.133,0.545,0.133},
    stringstyle=\color[rgb]{0.8,  0.1,  0.1},
}



% .:: Kopf- und Fußzeile
% ======================================================================
\usepackage{fancyhdr}
\pagestyle{fancy}
\fancyhf{}

   \fancyfoot[C]{von Hendrik Böttcher und Emanuel Regnath}
   \renewcommand{\headrulewidth}{0.0pt} %obere Linie ausblenden
   \renewcommand{\footrulewidth}{0.1pt} %obere Linie ausblenden

   \fancyfoot[R]{Stand: \today \ um \thistime \ Uhr \qquad \thepage}
   \fancyfoot[L]{Homepage: www.latex4ei.de -- Fehler bitte sofort an info@latex4ei.de}
	

% Dokumentbeginn
% ======================================================================
\begin{document}



% Aufteilung in Spalten
\vspace{-3mm}
\begin{multicols}{4}
	\fstitle{MATLAB}
% -------------------------------------------
% | 		MATLAB					|
% ~~~~~~~~~~~~~~~~~~~~~~~~~~~~~~~~~~~~~~~~~~~
%=======================================================================

\section{Wichtige Befehle}
\sectionbox{
\subsection{Standardbefehle}
\begin{tabular*}{\columnwidth}{@{\extracolsep\fill}ll@{}}
\ctrule
Befehl & Funktion\\ \cmrule
save(\textit{filename, variable}) & speichert \textit{variable} in matfile\\
load(\textit{filename}) & lädt Variable aus matfiel \\
clear \textit{variable} & löscht \textit{variable}\\
clear all & löscht alle Variablen im Workspace\\
clc & löscht Inhalt des Kommandofensters\\
doc \textit{expression} & Hilfedatei zu \textit{expression}\\
help \textit{expression} & Kurzhilfe zu \textit{expression} \\
\cbrule
\end{tabular*}
}

\sectionbox{
\subsection{Datentypkonvertierung (Karsten)}
\begin{tabular*}{\columnwidth}{@{\extracolsep\fill}ll@{}}
\ctrule
Befehl & Funktion \\\cmrule
double(\textit{array}) & Umwandlung von \textit{array} in double\\
\cbrule
\end{tabular*}
}

\sectionbox{
\subsection{Allgemeine Rechenoperationen}
\begin{tabular*}{\columnwidth}{@{\extracolsep\fill}ll@{}}
\ctrule
Befehl & Funktion \\\cmrule
mod(x,y) & x modulo y (immer positiv)\\
rem(x,y) & x modulo y (vorzeichenabhängig)\\
sqrt(x) & $ \sqrt{x}$\\
floor(x) & Abrunden auf Integer\\
ceil(x) & Aufrunden auf Integer\\
sum(x) & Summe über Werte des Vektors x\\
prod(x) & Produkt über Werte des Vektors x\\
min(x) & kleinster Wert des Vektors x\\
max(x) & größter Wert des Vektors x\\
all(x) & 1 für keine 0 in Vektor x\\
any(x) & 1 für eine Nicht-0 in Vektor x\\
\cbrule
\end{tabular*}
}

\section{Trigonometrische Funktionen}
\sectionbox{
\begin{tabular*}{\columnwidth}{@{\extracolsep\fill}ll@{}}
\ctrule
Befehl & Funktion \\\cmrule
sin(x) , cos(x), tan(x) & x in Bogenmaß!\\
sind(x), cosd(x), tand(x) & x in Grad!\\
asin(x), acos(x), atan(x) & Arcusfunktionen (Rad)\\
asind(x), acosd(x), antans(x) & Arcusfunktionen (Grad)\\
\cbrule
\end{tabular*}
}


\section{Komplexe Zahlen}
\sectionbox{
\begin{tabular*}{\columnwidth}{@{\extracolsep\fill}ll@{}}
\ctrule
Befehl & Funktion \\\cmrule
complex(a,b) & $a+jb$ \\
real(z) & Realteil von z\\
imag(z) & Imaginärteil von z\\
abs(z) & Betrag/Komplexe Amplitude von z\\
angle(z) & Phase von z\\
conj(z) & konjugiert komplex von z\\
\cbrule
\end{tabular*}
}

\section{Matrizzenrechnung}
\sectionbox{
\subsection{Rechenoperationen}
\begin{tabular}{l|l}
Befehl & Funktion \\\mrule
{[}a b c{]} & Zeilenvektor\\
{[}a; b; c{]} & Spaltenvektor\\
{[}a b c; d e f; g h i{]} & 3x3-Matrix\\
inv($\ma A$) & inverse Matrix von $\ma A$\\
$\ma A$' & $\ma A^\top$\\
$\ma A\setminus \vec b $ & löst $\ma A \vec x= \vec b$\\
$\ma A$(m,n) & Element $\ma A_{m,n}$\\
$\ma A$(m,:) & m. Zeile\\
$\ma A$(:,n) & n. Spalte\\
find($\ma A$) & lokalisiert Nicht-Null-Elemente (Indizes)\\
det($\ma A$) & Determinante von A \\
a:b:c & Vektor von a bis c mit Schrittweite b\\
linspace(a,b,n) & n Werte im gleichen Abstand von a bis b\\
norm(x) & eukl. Norm des Vektors x\\
$[\ma L \ma R \ma P ] = \text{lu}(\ma A)$ &(LR-) Zerlegung von A in Dreiecksmatrizen \\
$[\ma Q \ma R] = \text{qr}(\ma A)$& QR-Zerlegung von A\\
\end{tabular}

Komponentenweises Rechnen durch einen Punkt vor einem Operator\\
Bsp: $\ma A$.\^{}2 quadriert jedes Element der Matrix $\ma A$\\
Inlinefunktion: @(x)(f(x))\\
}

\sectionbox{
\subsection{Spezielle Matrizzen}
\begin{tabular}{l|l}
Befehl & Funktion \\\mrule
eye(m,n) & mxn Einheitsmatrix\\
zeros(m,n) & mxn 0-Matrix\\
ones(m,n) & mxn 1-Matrix\\
diag([a b]) & Diagonalmatrix mit [a b] auf Diagonale\\
rand(m,n) & mxn Zufallsmatrix (Werte: 0-1)\\
randi(imax,m,n) & integer Zufallsmatrix mit max. imax\\

\end{tabular}
}

\sectionbox{
\section{Schleiflab}
\begin{tabular}{l|l}
while: & for: \\ \trule
	\emph{while} expression & \emph{for} i=0:1:20\\
	statements & statements \\
	\emph{end} & \emph{end}\\
\end{tabular}\\
Schleife vorzeitig verlassen mit break
}


%=======================================================================
\section{Plot}
%=======================================================================

	\subsection{2D Plots}
	\begin{lstlisting}
figure(1);				% new figure
clf;					% clear old figures
plot(x, y, 'k');		% plot y(x) in black 'k'
hold on;				% more plots in same figure
plot(x, z, 'ro')		% plot z(x) in red circles
legend('y', 'z')		% names of plots
hold off;
	\end{lstlisting}


\sectionbox{
	\subsection{3D Plots}
	\begin{tabular}{l|l}
	Befehl & Funktion \\\mrule
	plot3($\vec x,\vec y,z$) & 3D-Plot mit Vektor x, y und z\\
	$[\ma X,\ma Y]$ = meshgrid($\vec x,\vec y$) & Erzeugt lineare Matrizen\\
	mesh($\vec x,\vec y,\ma Z$) & Skalarfeldplot\\
	surf($\vec x,\vec y,\ma Z,\ma C$) & Oberflächenplott mit Farbmatrix C\\
	\end{tabular}\\

	Beispiel:\\
	Laufparameter t: t=1:pi/10:10pi;\\
	plot3: plot3(sin(t),cos(t),t)\\
}


%=======================================================================
\section{Filter}
%=======================================================================





\sectionbox{
\section{Bildbearbeitung}
\begin{tabular}{l|l}
Befehl & Funktion \\\mrule
$\ma B$ = imread(\textit{url / filename}) & Bild einlesen\\
imshow($\ma B$) & Bild $\ma A$ anzeigen\\
\end{tabular}
}

\sectionbox{
\subsection{Filterung von Bildern}
\begin{tabular}{l|l}
Befehl & Funktion \\\mrule
conv2(Bild,$\ma F$,Parameter) & Faltung von Bildes mit $\ma F$\\
\end{tabular}

\begin{tabular}{l|l}
Parameter aus conv2 & Funktion \\\mrule
'same' & gleiche Größe wie Bild\\
\end{tabular}
}


%=======================================================================
\section{Include Matlab Plots in LaTeX}
%=======================================================================










% Ende der Spalten
\end{multicols}

% Dokumentende
% ======================================================================
\end{document}
